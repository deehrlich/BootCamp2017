 \documentclass[12pt]{article}

\usepackage[margin=1in]{geometry}
\usepackage{fancyhdr}
\usepackage{setspace}
\pagestyle{fancy}
\usepackage{amsmath, amsthm, amssymb, amsfonts, mathtools, xfrac,mathrsfs}
\usepackage[utf8]{inputenc}
\usepackage[english]{babel}
\usepackage{graphicx,dsfont}
\usepackage{braket, bm}

\everymath{\displaystyle}
\headheight=20pt


\newcommand{\N}{\mathbb{N}}
\newcommand{\Z}{\mathbb{Z}}
\newcommand{\Q}{\mathbb{Q}}
\newcommand{\R}{\mathbb{R}}
\newcommand{\E}{\mathrm{E}}
\newcommand{\Var} {\mathrm{Var}}
\newcommand{\Cov}{\mathrm{Cov}}
\newcommand{\F}{\mathbb{F}}

\DeclareMathOperator{\Tr}{Tr}

\def\mean#1{\left< #1 \right>}
 
\newcommand*\rfrac[2]{{}^{#1}\!/_{#2}}
 
\newenvironment{theorem}[2][Theorem]{\begin{trivlist}
\item[\hskip \labelsep {\bfseries #1}\hskip \labelsep {\bfseries #2}]}{\end{trivlist}}

\newenvironment{problem}[2][Problem]{\begin{trivlist}
\item[\hskip \labelsep {\bfseries #1}\hskip \labelsep {\bfseries #2}]}{\end{trivlist}}

\title{Homework}
\lhead{Econ OSM Lab}
\chead{Linearization}
\rhead{Dan Ehrlich}
 
\begin{document}

\begin{problem}{3.} 
We are given that 
$$E_t \big\{ F\tilde{X}_{t+1} +G\tilde{X}_{t} + H\tilde{X}_{t-1} + L\tilde{Z}_{t+1} + M\tilde{Z}_{t}\big\} = 0$$
$$\tilde{Z}_{t} = N \tilde{Z}_{t-1} + \epsilon_t$$
$$\tilde{X}_{t} = P \tilde{X}_{t-1} + Q\tilde{Z}_{t}$$
We express each term in the first equation as a function of the second and third and simply the resulting expression. Since the $E_t(\epsilon) = 0$, and all other terms are deterministic, the expectation can also be removed.
\begin{align*}
0 &= E_t \big\{ F\tilde{X}_{t+1} +G\tilde{X}_{t} + H\tilde{X}_{t-1} + L\tilde{Z}_{t+1} + M\tilde{Z}_{t}\big\} \\
& = F(P^2\tilde{X}_{t-1} + PQ\tilde{Z}_{t} + Q\tilde{Z}_{t+1}) + G( P \tilde{X}_{t-1} + Q\tilde{Z}_{t}) + H\tilde{X}_{t-1} + LN\tilde{Z}_{t} + M\tilde{Z}_{t}  \\
& = (FP^2+GP +H)\tilde{X}_{t-1} + (FPQ + GQ + LN + M)\tilde{Z}_{t} + FQ\tilde{Z}_{t+1}\\
 &= (FP^2+GP +H)\tilde{X}_{t-1} + (FPQ + GQ + LN + M + FQN)\tilde{Z}_{t}\\
&  = ((FP+G)P + H)\tilde{X}_{t-1} + ((FQ + L)N + (FP+G)Q +M)\tilde{Z}_{t}\
\end{align*}
\end{problem}

\begin{problem}{1 - 2, 4 - 11.} 
Refer to the appropriate python notebook.
\end{problem}

\end{document}

