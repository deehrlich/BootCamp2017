\documentclass[letterpaper,12pt]{article}
\usepackage{array}
\usepackage{threeparttable}
\usepackage{geometry}
\geometry{letterpaper,tmargin=1in,bmargin=1in,lmargin=1.25in,rmargin=1.25in}
\usepackage{fancyhdr,lastpage}
\pagestyle{fancy}
\lhead{}
\chead{}
\rhead{}
\lfoot{}
\cfoot{}
\rfoot{\footnotesize\textsl{Page \thepage\ of \pageref{LastPage}}}
\renewcommand\headrulewidth{0pt}
\renewcommand\footrulewidth{0pt}
\usepackage[format=hang,font=normalsize,labelfont=bf]{caption}
\usepackage{listings}
\lstset{frame=single,
  language=Python,
  showstringspaces=false,
  columns=flexible,
  basicstyle={\small\ttfamily},
  numbers=none,
  breaklines=true,
  breakatwhitespace=true
  tabsize=3
}
\usepackage{amsmath}
\usepackage{amssymb}
\usepackage{amsthm}
\usepackage{harvard}
\usepackage{setspace}
\usepackage{float,color}
\usepackage[pdftex]{graphicx}
\usepackage{hyperref}
\hypersetup{colorlinks,linkcolor=red,urlcolor=blue}
\theoremstyle{definition}
\newtheorem{theorem}{Theorem}
\newtheorem{acknowledgement}[theorem]{Acknowledgement}
\newtheorem{algorithm}[theorem]{Algorithm}
\newtheorem{axiom}[theorem]{Axiom}
\newtheorem{case}[theorem]{Case}
\newtheorem{claim}[theorem]{Claim}
\newtheorem{conclusion}[theorem]{Conclusion}
\newtheorem{condition}[theorem]{Condition}
\newtheorem{conjecture}[theorem]{Conjecture}
\newtheorem{corollary}[theorem]{Corollary}
\newtheorem{criterion}[theorem]{Criterion}
\newtheorem{definition}[theorem]{Definition}
\newtheorem{derivation}{Derivation} % Number derivations on their own
\newtheorem{example}[theorem]{Example}
\newtheorem{exercise}[theorem]{Exercise}
\newtheorem{lemma}[theorem]{Lemma}
\newtheorem{notation}[theorem]{Notation}
\newtheorem{problem}[theorem]{Problem}
\newtheorem{proposition}{Proposition} % Number propositions on their own
\newtheorem{remark}[theorem]{Remark}
\newtheorem{solution}[theorem]{Solution}
\newtheorem{summary}[theorem]{Summary}
\newcommand{\R}{\mathbb{R}}
\usepackage{mathrsfs}  
%\numberwithin{equation}{section}
\bibliographystyle{aer}
\newcommand\ve{\varepsilon}
\newcommand\boldline{\arrayrulewidth{1pt}\hline}


\begin{document}

\begin{flushleft}
  \textbf{\large{Problem Set \#2 Part 1}} \\
  OSM Lab, Professor Stachurski  \\
  Dan Ehrlich
\end{flushleft}

\vspace{3mm}

\noindent\textbf{Problem 1}
We first need to show that the set of complete and bounded functions $(\mathscr{C}, \|\cdot\|_{\sup}) $ is complete. We now show that $T$ is a contraction mapping on $\mathscr{C}$. We have that for a specific $y \in \R_+$: 
\begin{equation*}
\begin{aligned}
& \big | Uw(y) - Uw'(y) \big |=  \\
&= \big| u(\sigma(y))+ \beta \int w(f(y - \sigma(y))z) \phi(dz)- u(\sigma(y))- \beta \int w'(f(y - \sigma(y))z) \phi(dz) \big| \\
&= \beta \big| \int [w(f(y - \sigma(y))z) -  w'(f(y - \sigma(y))z)] \phi(dz) \big|  \\
& \leq \beta \int \big | (w(f(y - \sigma(y))z) - w'(f(y - \sigma(y))z))\big |  \phi(dz)  \\
& \leq \beta \int \|w-w'\|_{\sup_{y \in \R_+} } \phi(dz) = \beta \|w-w'\|_{\sup_{y \in \R_+} }
\end{aligned}
\end{equation*}
Taking the $\sup$ over $y \in \R_+$ gives us the desired result that $$\|Uw - Uw'\|_{\sup_{y \in \R_+} } \leq \beta \|w-w'\|_{\sup_{y \in \R_+} }$$ 
This proves that $U$ is a contraction mapping and there exists a unique fixed point solution. We now argue that the unique fixed point of $U$ in $\mathscr{C}$ is $v_{\sigma}$. We know that $v_{\sigma}$ is the expected  lifetime utility given that policy $\sigma(y)$ is used. If  $v'_{\sigma}(y')$ is the expected lifetime utility a period forward then  $v'_{\sigma}(y') = v_{\sigma}(y')$ because otherwise the individual's preferences must have changed. It is therefore a fixed point of the equation. \\\\
\noindent\textbf{Problem 2} Refer to the python notebook "DP\_Part2". 


\end{document}

