	 \documentclass[12pt]{article}

\usepackage[margin=1in]{geometry}
\usepackage{fancyhdr}
\usepackage{setspace}
\pagestyle{fancy}
\usepackage{amsmath, amsthm, amssymb, amsfonts, mathtools, xfrac,mathrsfs}
\usepackage[utf8]{inputenc}
\usepackage[english]{babel}
\usepackage{graphicx,dsfont}
\usepackage{braket, bm}

\everymath{\displaystyle}
\headheight=20pt
 
\newcommand{\N}{\mathbb{N}}
\newcommand{\Z}{\mathbb{Z}}
\newcommand{\Q}{\mathbb{Q}}
\newcommand{\R}{\mathbb{R}}
\newcommand{\E}{\mathrm{E}}
\newcommand{\Var} {\mathrm{Var}}
\newcommand{\Cov}{\mathrm{Cov}}

\DeclareMathOperator{\Tr}{Tr}

\def\mean#1{\left< #1 \right>}
 
\newcommand*\rfrac[2]{{}^{#1}\!/_{#2}}
 
\newenvironment{theorem}[2][Theorem]{\begin{trivlist}
\item[\hskip \labelsep {\bfseries #1}\hskip \labelsep {\bfseries #2}]}{\end{trivlist}}

\newenvironment{problem}[2][Problem]{\begin{trivlist}
\item[\hskip \labelsep {\bfseries #1}\hskip \labelsep {\bfseries #2}]}{\end{trivlist}}

\title{Homework}
\lhead{Math OSM Lab}
\chead{Homework 2 - 7/3/17}
\rhead{Dan Ehrlich}
 
\begin{document}

\begin{problem}{1.} We have that $\| x + y \|^2 = \|x\|^2 + \|y\|^2 + 2\| x\|\|y\|\cos(\theta)$ and $\| x - y \|^2 = \|x\|^2 + \|y\|^2 - 2\| x\|\|y\|\cos(\theta)$ where $\cos(\theta) = \frac{\langle x,y \rangle}{\| x\|\|y\|}$ which is proven in the textbook. 
\begin{itemize}
\item [(i)]  $\frac{1}{4}(\| x + y \|^2 - \| x - y \|^2 )= \frac{1}{4} (\|x\|^2 + \|y\|^2 + 2\| x\|\|y\|\cos(\theta) - \|x\|^2 - \|y\|^2 + 2\| x\|\|y\|\cos(\theta)) = \frac{1}{4} (4\| x\|\|y\|\cos(\theta)) = \| x\|\|y\| \frac{\langle x,y \rangle}{\| x\|\|y\|} = \langle x,y \rangle$
\item [(ii)]$ \frac{1}{2}(\| x + y \|^2 + \| x - y \|^2) = \frac{1}{2}( \|x\|^2 + \|y\|^2 + 2\| x\|\|y\|\cos(\theta) + \|x\|^2 + \|y\|^2 - 2\| x\|\|y\|\cos(\theta)) =  \frac{1}{2}(2(\|x\|^2 + \|y\|^2)) = \|x\|^2 + \|y\|^2$
\end{itemize}
\end{problem}

\begin{problem}{2.} 
\begin{equation*}
\begin{aligned}
& \frac{1}{4}(\| x + y \|^2 - \| x - y \|^2 + i\| x - iy \|^2 - i\| x + iy \|^2) = \langle x,y \rangle + \frac{1}{4}( i\| x - iy \|^2 - i\| x + iy \|^2) = \\
& = \langle x,y \rangle + \frac{1}{4}( -i( ||x||^2 +\langle x, iy \rangle +\langle  iy ,x\rangle +||y||^2  - ||x||^2 +\langle x, iy \rangle +\langle  iy ,x\rangle -||y||^2)) = \\
& = \langle x,y \rangle  + \frac{i}{4}(2i\langle x, y \rangle  - 2i\langle y,x \rangle ) = \langle x,y \rangle
\end{aligned}
\end{equation*}
\end{problem}

\begin{problem}{3.} \hfill
\begin{itemize}
\item [(i)] Let $f(x) = x$ and $g(x) = x^5$. Then 
$$\theta = \cos^{-1} \big(\frac{\langle f,g \rangle}{\| f\|\|g\|} \big )= \cos^{-1}\big( \frac{\int_0^1 x^6 dx}{ \sqrt{\int_0^1 x^2 dx}\sqrt{\int_0^1 x^{10}dx}}\big) =\cos^{-1} \big(\frac{1/7}{\sqrt{1/3}\sqrt{1/11}}\big) = 0.608$$
\item [(ii)] Let $f(x) = x^2$ and $g(x) = x^4$. Then 
$$\theta = \cos^{-1}\big( \frac{\int_0^1 x^6 dx}{ \sqrt{\int_0^1 x^4 dx}\sqrt{\int_0^1 x^{8}dx}}\big) =\cos^{-1} \big(\frac{1/7}{\sqrt{1/5}\sqrt{1/3}}\big) = 0.984$$
\end{itemize}
\end{problem}

\begin{problem}{8.}  \hfill
\begin{itemize}
\item [(i)] 
$$\langle \cos(t), \sin(t) \rangle  = \frac{1}{\pi} \int_{-\pi}^{\pi} \cos(t) \sin(t) dt = \frac{1}{\pi} * 0 = 0 $$
$$\langle \cos(t), \cos(2t) \rangle = \frac{1}{\pi} \int_{-\pi}^{\pi} \cos(t) \cos(2t) dt = 0 $$
$$\langle \cos(t), \sin(2t) \rangle = \frac{1}{\pi} \int_{-\pi}^{\pi} \cos(t) \sin(2t) dt = 0 $$
$$\langle \cos(t), \cos(t) \rangle = \frac{1}{\pi} \int_{-\pi}^{\pi} \cos(t) \cos(t) dt = 1 $$
$$\langle \sin(t), \cos(2t) \rangle = \frac{1}{\pi} \int_{-\pi}^{\pi} \sin(t) \cos(2t) dt = 0 $$
$$\langle \sin(t), \sin(2t) \rangle = \frac{1}{\pi} \int_{-\pi}^{\pi} \sin(t) \sin(2t) dt = 0 $$
$$\langle \sin(t), \sin(t) \rangle = \frac{1}{\pi} \int_{-\pi}^{\pi} \sin(t) \sin(t) dt = 1 $$
$$\langle \cos(2t), \sin(2t) \rangle = \frac{1}{\pi} \int_{-\pi}^{\pi} \cos(2t) \sin(2t)dt = 0 $$
$$\langle \cos(2t), \cos(2t) \rangle = \frac{1}{\pi} \int_{-\pi}^{\pi} \cos(2t) \cos(2t) dt = 1 $$
$$\langle \sin(2t), \sin(2t) \rangle = \frac{1}{\pi} \int_{-\pi}^{\pi} \sin(2t) \sin(2t) dt = 1$$
\item [(ii)] $$\|t\| = \sqrt{\frac{1}{\pi} \int_{-\pi}^{\pi} t^2 dt}  = \sqrt{\frac{2\pi^3}{3}}$$
\item [(iii)] 
\begin{equation*}
\begin{aligned}
&\textit{proj}_X(\cos(3t)) = \langle \cos(t), \cos(3t) \rangle + \langle \sin(t), \cos(3t) \rangle + \langle \cos(2t), \cos(3t) \rangle + \langle \sin(2t), \cos(3t) \rangle = \\
& = \frac{1}{\pi} \int_{-\pi}^{\pi} \cos(t) \cos(3t) dt  + \frac{1}{\pi} \int_{-\pi}^{\pi} \sin(t) \cos(3t) dt + \frac{1}{\pi} \int_{-\pi}^{\pi} \cos(2t) \cos(3t) dt + \\
& + \frac{1}{\pi} \int_{-\pi}^{\pi} \sin(2t) \cos(3t) dt  = 0 + 0 + 0 + 0 = 0
\end{aligned}
\end{equation*}
\item [(iv)] 
\begin{equation*}
\begin{aligned}
&\textit{proj}_X(t) = \frac{1}{\pi} \int_{-\pi}^{\pi} t\cos(t) dt  + \frac{1}{\pi} \int_{-\pi}^{\pi} t\sin(t)dt + \frac{1}{\pi} \int_{-\pi}^{\pi} t\cos(2t)dt + \frac{1}{\pi} \int_{-\pi}^{\pi} t\sin(2t) dt \\
&= \frac{1}{\pi}(0 + 2\pi +0 - \pi) = 1
\end{aligned}
\end{equation*}
\end{itemize}
\end{problem}

\begin{problem}{9.} Let $x = (x_1, x_2)$ and let $y = (y_1, y_2)$. Then $R_\theta x = (\cos(\theta)x_1-\sin(\theta)x_2, \sin(\theta)x_1 + \cos(\theta)x_2)$ and $R_\theta y = (\cos(\theta)y_1-\sin(\theta)y_2, \sin(\theta)y_1 + \cos(\theta)y_2)$. We now expand: 
\begin{equation*}
\begin{aligned}
\langle R_\theta x, R_\theta y \rangle = & (\cos(\theta)x_1-\sin(\theta)x_2)*(\cos(\theta)y_1-\sin(\theta)y_2) + \\& (\sin(\theta)x_1 + \cos(\theta)x_2)*(\sin(\theta)y_1 + \cos(\theta)y_2)
= \\ 
& \cos(\theta)^2x_1y_1 - \sin(\theta)\cos(\theta)x_2y_1 - \cos(\theta)\sin(\theta)x_1y_2 + \sin(\theta)^2x_2y_2 + \\
&  \sin(\theta)^2x_1y_1 + \cos(\theta)^2x_2 y_2 + \cos(\theta)\sin(\theta)x_1y_2 + \cos(\theta)\sin(\theta)x_2y_1 = \\
& (x_1y_1 + x_2y_2)( \cos(\theta)^2 + \sin(\theta)^2)  =  x_1y_1 + x_2y_2 = \langle x, y \rangle
\end{aligned}
\end{equation*}
\end{problem}

\begin{problem}{10.} Let $Q \in M_n(\mathbb{F})$ be an orthonormal matrix. 
\begin{itemize}
\item [(i)] $(\Rightarrow)$ Let $Q \in M_n(\mathbb{F})$ be an orthonormal matrix. Then $\langle Qx, Qy \rangle  = \langle x,y \rangle$ which expanding gives $\langle Qx, Qy \rangle = x^HQ^HQy = x^Hy$. This implies that $Q^HQ = I$. By Proposition 3.2.12, since $Q$ is an orthonormal operator and $\mathbb{F}^n$ is finite dimensional, $Q$ is invertible. Since inverses are unique,  $Q^{-1}  = Q^H$ so $Q^HQ = QQ^H = I$. \\
$(\Leftarrow)$ Let $Q^HQ = QQ^H = I$. Then $\langle Qx, Qy \rangle = x^HQ^HQy = x^HIy = x^Hy = \langle x,y \rangle$.

\item [(ii)] This follows directly from part $i$. $\|Qx\| = \sqrt{\langle Qx, Qx \rangle} = \sqrt{\langle x, x \rangle} = \|x\|$. 
\item [(iii)] By part $i$ we have $Q^{-1}  = Q^H$. Then $\langle Q^Hx, Q^Hy \rangle = x^HQQ^Hy = x^HIy = x^Hy = \langle x,y \rangle$.
\item [(iv)] Consider $\langle Qe_i, Qe_j \rangle = \langle e_i,e_j \rangle = \delta_{ij}$ which is the Kronecker delta. $Qe_i$ is column $i$ of matrix $Q$, so the dot product of column $i$ with itself is 1 and 0 when $i \neq j$, implying the columns of an orthonormal matrix are orthonormal.  
\item [(v)] $\det(QQ^H) = \det(I) = 1$. Since $\det(Q) = \det(Q^H)$ we have $\det(Q)^2= 1$ so $\det(Q)= 1$. The converse is not necessarily true. Consider: 
\[ A= 
\begin{bmatrix}
    1   & 1 & 0 \\
    0   & 1& 0 \\
    0   & 0 &1 
\end{bmatrix}
\]
The matrix $A$ is upper triangular and therefore has $\det(A) = 1$ but is not orthonormal because column 2 dot product itself is not equal to 1 which needs to be true by part $(iv)$. 
\item [(vi)] Let $Q_1, Q_2$ be orthonormal. Then $\langle Q_1Q_2x, Q_1Q_2y \rangle = x^HQ_2^HQ_1^HQ_1Q_2y = x^HQ_2^HIQ_2y = x^HQ_2^HQ_2y = x^Hy = \langle x ,y \rangle$ so the product $Q_1Q_2$ is orthonormal. 
\end{itemize}
\end{problem}

\begin{problem}{11.} Applying the Gram-Schmidt orthonormalization process to a collection of linearly dependent vectors we are ultimately forced to divide by zero when trying to form the orthonormal vector for the first dependent vector in the set. We can see this by setting WLOG the first dependent vector to be the second vector in the set. Then 
$$q_1 = \frac{x_1}{\|x_1\|}$$
$$q_2 = \frac{x_2 - p_2}{\|x_2 - p_2\|}$$
where $p_2 = \langle x_2, q_1 \rangle q_1$. However, assuming $x_2 = a_1 x_1$, we have that 
$$p_2 =  \langle a_1 x_1,  \frac{x_1}{\|x_1\|} \rangle  \frac{x_1}{\|x_1\|} = \langle   \frac{x_1}{\|x_1\|},  \frac{x_1}{\|x_1\|} \rangle   a_1x_1 = a_1x_1 = x_2$$
so $x_2 - p_2 = 0$. 
\end{problem}

\begin{problem}{16.} 
\end{problem}

\begin{problem}{17.} 
Let $A = \widehat{Q}\widehat{R}$ be reduced QR decomposition. Then the system $A^HAx = A^Hb$ can be rewritten as 
\begin{equation*}
\begin{aligned}
&(\widehat{Q}\widehat{R})^H(\widehat{Q}\widehat{R})x = (\widehat{Q}\widehat{R})^Hb\\
& \Leftrightarrow \widehat{R}^H\widehat{Q}^H\widehat{Q}\widehat{R}x = \widehat{R}^H\widehat{Q}^Hb \\ &\Leftrightarrow \widehat{R}^H\widehat{R}x = \widehat{R}^H\widehat{Q}^Hb \\
&\Leftrightarrow \widehat{R}x = \widehat{Q}^Hb 
\end{aligned}
\end{equation*}
\end{problem}

\begin{problem}{23.} 
We have that $\|x\| = \|x - y + y\| \leq \|x-y\| + \|y\|$ so  $\|x\| - \|y\|  \leq \|x-y\| $. Similarly, $\|y\| - \|x\|  \leq \|x-y\|$ since $\|x-y\| = \|y-x\|$. Putting these together implies that $|\|x\| - \|y\|| \leq \|x-y\|$ because if $\|x\| > \|y\|$ then $|\|x\| - \|y\|| =  \|x\| - \|y\| $ and else $|\|x\| - \|y\|| = \|y\| - \|x\|$.
\end{problem}

\begin{problem}{24.} \hfill
\begin{itemize}
\item [(i)] 
\item [(ii)] 
\item [(iii)]
\end{itemize} 
\end{problem}

\begin{problem}{26.} 
\end{problem}


\begin{problem}{28.} 
\end{problem}


\begin{problem}{29.} 
\end{problem}


\begin{problem}{30.} 
\end{problem}

\begin{problem}{37.} 
\end{problem}

\begin{problem}{47.} Let $P = A(A^HA)^{-1}A^H$.
\begin{itemize}
\item [(i)]  Since $(A^HA)^{-1}A^HA= I$ we have that $$P^2 =PP= A(A^HA)^{-1}A^HA(A^HA)^{-1}A^H = A(A^HA)^{-1}A^H = P$$
\item [(ii)] $P^H = (A(A^HA)^{-1}A^H)^H = (A^H)^H ((A^HA)^{-1})^H A^H = A ((A^HA)^H)^{-1}A^H = A(A^HA)^{-1}A^H = P$
\item [(iii)] 
\end{itemize} 
\end{problem}

\end{document}




