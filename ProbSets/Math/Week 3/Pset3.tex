	 \documentclass[12pt]{article}

\usepackage[margin=1in]{geometry}
\usepackage{fancyhdr}
\usepackage{setspace}
\pagestyle{fancy}
\usepackage{amsmath, amsthm, amssymb, amsfonts, mathtools, xfrac,mathrsfs}
\usepackage[utf8]{inputenc}
\usepackage[english]{babel}
\usepackage{graphicx,dsfont}
\usepackage{braket, bm}

\everymath{\displaystyle}
\headheight=20pt


\newcommand{\N}{\mathbb{N}}
\newcommand{\Z}{\mathbb{Z}}
\newcommand{\Q}{\mathbb{Q}}
\newcommand{\R}{\mathbb{R}}
\newcommand{\E}{\mathrm{E}}
\newcommand{\Var} {\mathrm{Var}}
\newcommand{\Cov}{\mathrm{Cov}}
\newcommand{\F}{\mathbb{F}}

\DeclareMathOperator{\Tr}{Tr}

\def\mean#1{\left< #1 \right>}
 
\newcommand*\rfrac[2]{{}^{#1}\!/_{#2}}
 
\newenvironment{theorem}[2][Theorem]{\begin{trivlist}
\item[\hskip \labelsep {\bfseries #1}\hskip \labelsep {\bfseries #2}]}{\end{trivlist}}

\newenvironment{problem}[2][Problem]{\begin{trivlist}
\item[\hskip \labelsep {\bfseries #1}\hskip \labelsep {\bfseries #2}]}{\end{trivlist}}

\title{Homework}
\lhead{Math OSM Lab}
\chead{Homework 2 - 7/3/17}
\rhead{Dan Ehrlich}
 
\begin{document}

\begin{problem}{2.}
Let $V = span(\{1,x,x^2\})$ and $D$ is the derivative operator $D: V \to V$ such that $D[p](x) = p'(x)$. In the Chapter 3 exercises we showed that \[
   D=
  \left[ {\begin{array}{ccc}
   0 & 1 & 0\\
   0 & 0 & 2\\
   0 & 0 & 0\\
  \end{array} } \right]
\]
Then $\det(D - \lambda I) = (-\lambda)^3 = 0$. Therefore the eigenvalue is $\lambda = 0$ with algebraic multiplicity 3 and geometric multiplicity of 0 since all the eigenvectors are in the form $(a, 0, 0)$.
 \end{problem}

\begin{problem}{4.} 
Let   \[ A=
  \left[ {\begin{array}{ccc}
   a & b\\
   c & d\\
  \end{array} } \right]
\]

\begin{itemize}
\item [(i)] If $A = A^H$ then 
\[ A=
  \left[ {\begin{array}{ccc}
   a & b\\
   b & d\\
  \end{array} } \right]
\]
Then $tr(A) = a + d$ and $\det(A) = ad-b^2$. $A$ has real eigenvalues if $tr(A)^2 - 4\det(A) \geq 0$ where $tr(A)^2 - 4\det(A)$ is the term under the square root  in the quadratic formula. We have that $tr(A)^2 - 4\det(A) = a^2 + d^2 + 2ad - 4ad  + 4b^2 = (a -d)^2 + 4b^2  \geq 0$ so $A$ has real eigenvalues.
\item [(ii)] If $A = -A^H$ then 
\[ A=
  \left[ {\begin{array}{ccc}
   0 & b\\
   -b & 0\\
  \end{array} } \right]
\]
Then $tr(A) = 0$ and $\det(A) = b^2$. We have that $tr(A)^2 - 4\det(A) = -4b^2  < 0$ so $A$ has imaginary eigenvalues.
\end{itemize}
\end{problem}

\begin{problem}{6.} Let the matrix $A$ be upper triangular. Then, since upper triangular matrices are closed under addition and subtraction, $A-\lambda I$ is also an upper triangular matrix with diagonal entires $a_{ii} - \lambda$. The determinant of an upper triangular matrix is the product of the diagonal entires so $p_A(\lambda) = \det( A-\lambda I) = \Pi_{i=1}^n  (a_{ii} - \lambda) = 0$. Therefore the eigenvalues of A are the values of $\lambda$ such that $a_{ii} - \lambda = 0 \forall i$ or equivalently  $a_{ii} = \lambda  \forall i$ where the $a_{ii}$ are the diagonal entires of A. 
\end{problem}

\begin{problem}{8.} Let $S = \{\sin(x), \cos(x), \sin(2x), \cos(2x) \}$ and $V = span(\{S\})$. 
\begin{itemize}
\item [(i)] $S$ is a basis for $V$ if $S$ spans V and is linearly independent. The first part follows from our assumptions. We now show the second. We proved in problem 3.8 that the set $S$ is orthonormal which implies that they are linearly independent and are hence a basis for $V$. 
\item [(ii)] The derivatives of the basis are $\{\cos(x), -\sin(x), 2\cos(2x), -2\sin(2x)\}$ respectively. Then \[D=
  \left[ {\begin{array}{cccc}
   0 & 1 & 0 & 0\\
   -1 & 0 & 0 & 0\\
   0 & 0 & 0 & 2\\
   0 & 0 & -2 & 0\\
  \end{array} } \right]
\]
\item [(iii)]
\end{itemize}
\end{problem}

\begin{problem}{13.} 
\end{problem}

\begin{problem}{15.} 
\end{problem}

\begin{problem}{16.} 
\end{problem}

\begin{problem}{18.} 
\end{problem}

\begin{problem}{20.} 
\end{problem}

\begin{problem}{24.} 
\end{problem}

\begin{problem}{25.}
\end{problem}

\begin{problem}{27.} 
\end{problem}

 
\begin{problem}{28.} 
\end{problem}


\begin{problem}{31.} 
\end{problem}


\begin{problem}{32.}
\end{problem}

\begin{problem}{33.} 
\end{problem}

\begin{problem}{36.} 
\end{problem}

\begin{problem}{38.} 
\end{problem}

\end{document}




