	 \documentclass[12pt]{article}

\usepackage[margin=1in]{geometry}
\usepackage{fancyhdr}
\usepackage{setspace}
\pagestyle{fancy}
\usepackage{amsmath, amsthm, amssymb, amsfonts, mathtools, xfrac,mathrsfs}
\usepackage[utf8]{inputenc}
\usepackage[english]{babel}
\usepackage{graphicx,dsfont}
\usepackage{braket, bm}

\everymath{\displaystyle}
\headheight=20pt


\newcommand{\N}{\mathbb{N}}
\newcommand{\Z}{\mathbb{Z}}
\newcommand{\Q}{\mathbb{Q}}
\newcommand{\R}{\mathbb{R}}
\newcommand{\E}{\mathrm{E}}
\newcommand{\Var} {\mathrm{Var}}
\newcommand{\Cov}{\mathrm{Cov}}
\newcommand{\F}{\mathbb{F}}

\DeclareMathOperator{\Tr}{Tr}

\def\mean#1{\left< #1 \right>}
 
\newcommand*\rfrac[2]{{}^{#1}\!/_{#2}}
 
\newenvironment{theorem}[2][Theorem]{\begin{trivlist}
\item[\hskip \labelsep {\bfseries #1}\hskip \labelsep {\bfseries #2}]}{\end{trivlist}}

\newenvironment{problem}[2][Problem]{\begin{trivlist}
\item[\hskip \labelsep {\bfseries #1}\hskip \labelsep {\bfseries #2}]}{\end{trivlist}}

\title{Homework}
\lhead{Math OSM Lab}
\chead{Homework 2 - 7/3/17}
\rhead{Dan Ehrlich}
 
\begin{document}

\begin{problem}{2.}
Let $V = span(\{1,x,x^2\})$ and $D$ is the derivative operator $D: V \to V$ such that $D[p](x) = p'(x)$. In the Chapter 3 exercises we showed that \[
   D=
  \left[ {\begin{array}{ccc}
   0 & 1 & 0\\
   0 & 0 & 2\\
   0 & 0 & 0\\
  \end{array} } \right]
\]
Then $\det(D - \lambda I) = (-\lambda)^3 = 0$. Therefore the eigenvalues  are $\lambda = 0$ with multiplicity 3. 
 \end{problem}

\begin{problem}{4.} \hfill
\begin{itemize}
\item [(i)] 
\item [(ii)]
\end{itemize}
\end{problem}

\begin{problem}{6.} \hfill
\begin{itemize}
\item [(i)] 
\item [(ii)]
\end{itemize}
\end{problem}

\begin{problem}{8.}
\end{problem}

\begin{problem}{13.} 
\end{problem}

\begin{problem}{15.} 
\end{problem}

\begin{problem}{16.} 
\end{problem}

\begin{problem}{18.} 
\end{problem}

\begin{problem}{20.} 
\end{problem}

\begin{problem}{24.} 
\end{problem}

\begin{problem}{25.}
\end{problem}

\begin{problem}{27.} 
\end{problem}

 
\begin{problem}{28.} 
\end{problem}


\begin{problem}{31.} 
\end{problem}


\begin{problem}{32.}
\end{problem}

\begin{problem}{33.} 
\end{problem}

\begin{problem}{36.} 
\end{problem}

\begin{problem}{38.} 
\end{problem}

\end{document}




