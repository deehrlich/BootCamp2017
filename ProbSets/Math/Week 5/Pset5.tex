 \documentclass[12pt]{article}

\usepackage[margin=1in]{geometry}
\usepackage{fancyhdr}
\usepackage{setspace}
\pagestyle{fancy}
\usepackage{amsmath, amsthm, amssymb, amsfonts, mathtools, xfrac,mathrsfs}
\usepackage[utf8]{inputenc}
\usepackage[english]{babel}
\usepackage{graphicx,dsfont}
\usepackage{braket, bm}

\everymath{\displaystyle}
\headheight=20pt


\newcommand{\N}{\mathbb{N}}
\newcommand{\Z}{\mathbb{Z}}
\newcommand{\Q}{\mathbb{Q}}
\newcommand{\R}{\mathbb{R}}
\newcommand{\E}{\mathrm{E}}
\newcommand{\Var} {\mathrm{Var}}
\newcommand{\Cov}{\mathrm{Cov}}
\newcommand{\F}{\mathbb{F}}

\DeclareMathOperator{\Tr}{Tr}

\def\mean#1{\left< #1 \right>}
 
\newcommand*\rfrac[2]{{}^{#1}\!/_{#2}}
 
\newenvironment{theorem}[2][Theorem]{\begin{trivlist}
\item[\hskip \labelsep {\bfseries #1}\hskip \labelsep {\bfseries #2}]}{\end{trivlist}}

\newenvironment{problem}[2][Problem]{\begin{trivlist}
\item[\hskip \labelsep {\bfseries #1}\hskip \labelsep {\bfseries #2}]}{\end{trivlist}}

\title{Homework}
\lhead{Math OSM Lab}
\chead{Homework 4 - 7/14/17}
\rhead{Dan Ehrlich}
 
\begin{document}

\begin{problem}{1.}
Let $S$ be a nonempty subset of $V$. Consider $conv(S)$ which is the set of all elements such that $\lambda_1x_1 + \cdots + \lambda_kx_k \in conv(S)$ where $x_i \in S, k \in \N$ and $\lambda_i \geq 0$ and $\lambda_1 + \cdots + \lambda_k = 1$. Consider then the case where $k=2$. We then have that $\lambda_1x + \lambda_2 y \in conv(S)$. Then by our assumptions, $\lambda_1 + \lambda_2 = 1$ so $\lambda \equiv \lambda_1 = 1- \lambda_2$. Then for any $x,y \in S$ we have $\lambda x + (1-\lambda)y \in conv(S)$ so $conv(S)$ is convex. 
\end{problem}

\begin{problem}{2.}\hfill
\begin{itemize}
\item[(i)] Let $P = \{x \in V | \langle a, x \rangle = b\} \in V$ be a hyperplane where $a \in V, a \neq 0, b\in \R$. Let $x, y \in P$ and consider the point $\lambda x + (1-\lambda)y$. Then $\langle a, \lambda x + (1-\lambda)y \rangle = \lambda \langle a, x \rangle + (1-\lambda)\langle a, y \rangle = \lambda b + (1-\lambda)b = b$ so $\lambda x + (1-\lambda)y \in P$ and $P$ is convex. 
\item[(ii)]Let $H = \{x \in V | \langle a, x \rangle \leq b\} \in V$ be a half-space where $a \in V, a \neq 0, b\in \R$. Let $x, y \in H$ and consider the point $\lambda x + (1-\lambda)y$. Then  $\langle a, \lambda x + (1-\lambda)y \rangle = \lambda \langle a, x \rangle + (1-\lambda)\langle a, y \rangle \leq \lambda b + (1-\lambda)b = b$ so $\lambda x + (1-\lambda)y \in H$ and $H$ is convex. 
\end{itemize}
\end{problem}


\begin{problem}{4.} \hfill
\begin{itemize}
\item[(i)]$\| x - y\|^2 = \langle x - y, x - y \rangle = \langle (x - p ) + (p- y) , (x - p ) + (p- y)\rangle = \| x - p\|^2 + \| p - y\|^2 + 2 \langle x - p, p - y \rangle$
\item[(ii)] Let $\langle x - p, p- y \rangle \geq 0$. Then if $y \neq p$ we have by part 1 that $\| x - y\|^2 = \| x - p\|^2 + \| p - y\|^2 + 2 \langle x - p, p - y \rangle$ where both the second and third terms on the right hand side are greater than $0$ so the left hand side is greater than the right hand side and $\| x - y\|^2 > \| x - p\|^2$. 
\item[(iii)] Let $z = \lambda y + (1-\lambda)p$ where $0 \geq \lambda \leq 1$.  Then $\| x - z\|^2 = \| x -  \lambda y -(1-\lambda)p\|^2 =  \langle x -  \lambda y - (1-\lambda)p, x -  \lambda y  - (1-\lambda)p \rangle = \langle (x - p) - \lambda (y - p) , (x - p) - \lambda (y - p)  \rangle=  \| x - p\|^2 +2 \lambda \langle x - p, y - p \rangle + \lambda^2 \| y - p\|^2$
\item[(iv)] Let $p$ be the projection of point $x$ onto $C$. Then by definition of the projection $\|x - p\|^2 \leq \|x - y\|^2,  \hspace{0.2cm} \forall y \in C$. We have from part (iii) that  $\| x - z\|^2  -  \| x - p\|^2 =  2 \lambda \langle x - p, y - p \rangle + \lambda^2 \| y - p\|^2$. Then by the definition of the projection $\| x - z\|^2  -  \| x - p\|^2 \geq 0$ so $0 \leq 2 \langle x - p, y - p \rangle + \lambda \| y - p\|^2$. 
\end{itemize}
\end{problem}

\begin{problem}{6.} 
Let $f: \R^n \to \R$ be a convex function and consider the set $C = \{ x \in \R^n | f(x) \leq c\} \subset \R^n$. Then for $x,y \in C$ and $0 \leq \lambda \leq 1$, since $f$ is convex, we have that $f(\lambda x + (1-\lambda)y) \leq \lambda f(x) + (1-\lambda)f(y) \leq \lambda c + (1-\lambda)c = c$. Therefore, $\lambda x + (1-\lambda)y \in C$ so $C$ is convex. 
\end{problem}

\begin{problem}{7.} 
Let $C$ be a convex set and let $f_1, \cdots, f_k$ be convex functions where $f_i:C \to \R$ and $\lambda_1, \cdots, \lambda_k \in \R^+$. Then define the function $f$ as $f(x) = \sum_{i=1}^{k} \lambda_if_i(x)$. Consider the points $x, y \in C$ and $0 \leq \lambda \leq 1$ so $\lambda x + (1-\lambda)y \in C$ since $C$ is convex . Then
\begin{align*}
 f(\lambda x + (1-\lambda)y) &= \sum_{i=1}^{k} \lambda_if_i(\lambda x + (1-\lambda)y) \\
 & \leq \sum_{i=1}^{k} \lambda_i (\lambda f_i(x) + (1-\lambda)f_i(y))   \\
  & =\sum_{i=1}^{k} \lambda_i \lambda f_i(x) + \sum_{i=1}^{k}  \lambda_i(1-\lambda)f_i(y) \\
  & =\lambda  \sum_{i=1}^{k} \lambda_i f_i(x) + (1-\lambda) \sum_{i=1}^{k}  \lambda_if_i(y) \\
  & = \lambda f(x) + (1- \lambda) f(y)
 \end{align*}
\end{problem}

\begin{problem}{13.} 
Let $f: \R^n \to \R$ be a convex function and bounded above. Assume by contradiction that $f$ is not constant. Then WLOG $\exists \hspace{0.1cm} x,y, z, \in \R^n$ such that $x <y $ and $f(x) < f(y)$. Additionally, let  $z \in (x,y)$. We have by convexity that $f(\lambda x + (1-\lambda) y) \geq \lambda f(x) + (1 - \lambda) f(y)$. If we let $\lambda  = \frac{z-y}{y-x}$ then $\lambda x + (1-\lambda) y = z$. Rewriting the above expression we have that $f(z) \geq \lambda f(x) + (1 - \lambda) f(y)$. If we let $y \to \infty$ then $f$ violates the condition of being bounded above, and must therefore be a constant function. 
\end{problem}

\begin{problem}{20.} 
Let $f: \R^n \to R$ be convex and let $-f$ also be convex. We have then that 
$$f(\lambda x + (1-\lambda) y) \geq \lambda f(x) + (1 - \lambda) f(y)$$
$$- f(\lambda x + (1-\lambda) y) \geq - (\lambda f(x) + (1 - \lambda) f(y))$$
These two conditions together imply that 
$$f(\lambda x + (1-\lambda) y) = \lambda f(x) + (1 - \lambda) f(y)$$
so f is linear and therefore affine. 
\end{problem}

\begin{problem}{21.} 
Let $x^*$ be the local minimizer for the problem
\begin{align*}
\min \hspace{1cm} & \phi \circ f(x) &\\
 \text{ s.t } \hspace{1cm}  & G(x) \leq 0& \\
&H(x) = 0&
\end{align*}
\end{problem}

\end{document}




